\chapter{Introdução}
\label{cap-introducao}

Ao lado da própria dificuldade em se mensurar objetivamente a usabilidade de um sistema, é comum em programas de software livre que haja pouco incentivo (ou interesse) nesse aspecto, dado que a prioridade do mesmo é a implementação das funcionalidades. Essa cultura leva o desenvolvedor a iniciar o projeto pelo código, deixando o design de interfaces em segundo plano~\cite{thomas2008}.
%

Métodos de usabilidade são utilizados por empresas especialistas em projeto de interação e, nos processos de desenvolvimento de projetos de software fechado, o que garante, em geral, uma melhor usabilidade quando comparados à maioria dos sistemas de software livre.
%

Esse cenário é um dos fatores que limita a expansão de uso e aceitação de sistemas livres, que com baixa usabilidade, perde-se também em confiança dos usuários. O uso desse tipo de sistema limita-se a usuários mais experientes em computação, o que resulta em perdas para usuários inexperientes. Muitos sistemas de software livre possuem código de qualidade, o que lhes confere algoritmos eficientes e de bom desempenho, pois foram produzidos, na maioria das vezes, por indivíduos motivados a resolver desafios ou problemas no sistema. Santos (\citeyear{santos2012}) ressalta que a perda de usuários potenciais devido à baixa usabilidade configura, portanto, um desperdício de recursos para a sociedade.

%
Já Nichols e Twidale (\citeyear{nichols2006}) afirmam que desenvolvedores de software livre não costumam notar a baixa usabilidade de diversos sistemas, por serem geralmente usuários experientes e acostumados a interfaces de baixo nível como a linha de comando, e em contrapartida, empresas que produzem software comercial muitas vezes possuem funcionários especialistas em usabilidade, raro em equipes de desenvolvimento de software livre.
%

Apesar desse cenário, o desenvolvimento de software livre vem mudando nos últimos anos com a crescente preocupação com a usabilidade, ainda que essa preocupação esteja normalmente limitada a projetos de grande visibilidade, geralmente patrocinados por grandes empresas. Porém, no âmbito das comunidades, a usabilidade raramente é considerada no processo \cite{nichols2006}.

%------------------------------------------------------------------------------%

\section{Objetivos}

Esta seção apresenta os objetivos gerais e específicos deste Trabalho de Conclusão de Curso (TCC).

\subsection{Objetivos Gerais}
Neste trabalho de conclusão de curso, temos o objetivo de implementar tarefas de usabilidade dentro do ciclo de desenvolvimento de um projeto de software livre, mantendo o responsável pela usabilidade inserido na equipe como implementador e ao final levantar os resultados da aplicação da usabilidade através de métodos de avaliação.

%------------------------------------------------------------------------------%

\subsection{Específicos}
Os objetivos específicos desse trabalho são:

\begin{enumerate}
\item Identificar as técnicas de usabilidade ágeis que possam ser utilizadas na comunidade de software livre.
\item Detalhar e definir uma técnica de usabilidade a ser aplicada.
\item Definir interação no projeto com a equipe de desenvolvimento.
\item Implementar práticas de usabilidade. 	
\item Avaliar e apresentar resultados da usabilidade do projeto para a equipe.
\end{enumerate}

%------------------------------------------------------------------------------%

\section{Organização do Trabalho}
No Capítulo \ref{cap-software-livre} encontra-se uma definição de software 
livre, suas liberdades e principais características; no Capítulo \ref{cap-metodos-ageis} uma breve descrição resumida dos seus princípios, práticas, 
principais métodos e sobre o manifesto ágil; no Capítulo \ref{cap-usabilidade} uma revisão bibliográfica sobre usabilidade, seus aspectos 
sobre a visão dos principais autores e terminologias associadas, após foi 
realizado uma revisão sistemática para levantar técnicas de usabilidade 
ágeis que possam ser aplicadas em comunidades de software livre e 
escolhida uma para detalhamento e uso no trabalho proposto, além da 
apresentação dos métodos de avaliação que serão utilizados para avaliar os 
resultados da aplicação da usabilidade; no Capítulo \ref{cap-estudo-caso} é 
apresentado o estudo de caso, a ferramenta Mezuro que se encontra em um 
segundo ciclo de desenvolvimento voltado para uma plataforma standalone que possibilita a implementação das tarefas de usabilidade. 
