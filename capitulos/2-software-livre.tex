\chapter{Software Livre}
\label{cap-software-livre}

%
O software, em um sistema computacional, é o componente que contém o conhecimento relacionado aos problemas a que a computação se aplica, contendo diversos aspectos que ultrapassam questões técnicas~\cite{meirelles2013}
, por exemplo:
\begin{itemize}
\item O processo de desenvolvimento do software;
\item Os mecânismos econômicos (gerenciais, competitivos, sociais, cognitivos etc.) que regem esse desenvolvimento e seu uso;
\item O relacionamento entre desenvolvedores, fornecedores e usuários do software;
\item Os aspectos éticos e legais relacionados ao software
\end{itemize} 

O que define e diferencia o software livre de software proprietário vai do entendimento desses quatro pontos dentro do que é conhecido como \textit{ecossistema do software livre}~\cite{meirelles2013}.
O princípio básico desse ecossistema refere à liberdade dos usuários de executar, copiar, distribuir, estudar, mudar e melhorar o software. Estas estão definidas nas quatro liberdades para os usuários do software:

\begin{itemize}
\item A liberdade de executar o programa, para qualquer propósito (liberdade no. 0);
\item A liberdade de estudar como o programa funciona, e adaptá-lo para as suas necessidades (liberdade no. 1). Aceso ao código-fonte é um pré-requisito para esta liberdade;
\item A liberdade de redistribuir cópias de modo que você possa ajudar ao seu próximo (liberdade no. 2);
\item A liberdade de aperfeiçoar o programa, e liberar os seus aperfeiçoamentos, de modo que toda a comunidade se beneficie (liberdade no. 3). Acesso ao código-fonte é um pré-requisito para esta liberdade.
\end{itemize}

Um programa é software livre se os usuários têm todas essas liberdades. Assim, você deve ser livre para redistribuir cópias, seja com ou sem modificações, seja de graça ou cobrando uma taxa pela distribuição, para qualquer um em qualquer lugar.

Você deve também ter a liberdade de fazer modificações e usá-las privativamente no seu trabalho ou lazer, sem nem mesmo mencionar que elas existem. Se você publicar as modificações, você não deve ser obrigado a avisar a ninguém em particular, ou de qualquer forma particular.

A liberdade de executar o programa significa a liberdade para qualquer tipo de pessoa física ou jurídica utilizar o software em qualquer tipo de sistema computacional, para qualquer tipo de trabalho em geral e finalidade, sem que seja necessário comunicar sobre isso com o desenvolvedor ou a qualquer outra entidade em especial.

A liberdade de redistribuir cópias deve incluir formas binárias ou executáveis do programa, assim como o código-fonte, tanto para as versões modificadas.

A liberdade de acesso ao código-fonte do programa. Portanto, acesso ao código-fonte é uma condição necessária ao software livre ~\cite{gnu2013}.

%
Essa liberdade dentro do software livre é dita por Richard Stallman como a principal vantagem. Software não-livre é ruim porque nega sua liberdade. Então, perguntar sobre as vantagens práticas do software livre é como perguntar sobre as vantagens práticas de não estar algemado ~\cite{stallman2009}.

%
O GNU quis dar aos utilizadores a liberdade, não só para ser popular. Contudo precisava usar termos de distribuição que impediriam software livre de ser transformado em software proprietário. O método usado foi chamado de copyleft. Copyleft é um método geral para fazer um software livre e exige que todas as versões modificadas e estendidas do programa sejam também.Ele utiliza lei de direitos autorais, mas veio para servir como oposto de sua finalidade usual: ao invés de um meio de privatizar o software, torna-se um meio de manter o software livre~\cite{stallman2009}

