\chapter{Software Livre}
\label{cap:midias-sociais}
%/ref{cap:midias-sociais}

\section{SL}

%
De acordo com Richard Stallmann  um programa é um software livre se os usuários possuírem 4 liberdades essenciais:
\begin{itemize}
\item A liberdade de executar o programa, para qualquer propósito (liberdade 0).
\item A liberdade de estudar como o programa funciona, e adaptá-lo às suas necessidades (liberdade 1). Para tanto, acesso ao código-fonte é um pré-requisito.
\item A liberdade de redistribuir cópias de modo que você possa ajudar ao próximo (liberdade 2).
\item A liberdade de distribuir cópias de suas versões modificadas a outros (liberdade 3).
\end{itemize}

Desta forma, você pode dar a toda comunidade a chance de beneficiar de suas mudanças. Para tanto, acesso ao código-fonte é um pré-requisito.