\chapter{Software Livre}
\label{cap-software-livre}

\section{SL}

%
O Termo "Software livre" se refere à liberdade dos usuários de executar, copiar, distribuir, estudar, mudar e melhorar o software. Estas estão definidas nas quatro liberdades para os usuários do software:

\begin{itemize}
\item A liberdade de executar o programa, para qualquer propósito (liberdade no. 0);
\item A liberdade de estudar como o programa funciona, e adaptá-lo para as suas necessidades (liberdade no. 1). Aceso ao código-fonte é um pré-requisito para esta liberdade;
\item A liberdade de redistribuir cópias de modo que você possa ajudar ao seu próximo (liberdade no. 2);
\item A liberdade de aperfeiçoar o programa, e liberar os seus aperfeiçoamentos, de modo que toda a comunidade se beneficie (liberdade no. 3). Acesso ao código-fonte é um pré-requisito para esta liberdade.
\end{itemize}

Um programa é software livre se os usuários têm todas essas liberdades. Assim, você deve ser livre para redistribuir cópias, seja com ou sem modificações, seja de graça ou cobrando uma taxa pela distribuição, para qualquer um em qualquer lugar. Ser livre para fazer essas coisas significa (entre outras coisas) que você não tem que pedir ou pagar pela permissão para fazê-lo.

Você deve também ter a liberdade de fazer modificações e usá-las privativamente no seu trabalho ou lazer, sem nem mesmo mencionar que elas existem. Se você publicar as modificações, você não deve ser obrigado a avisar a ninguém em particular, ou de qualquer forma particular.

A liberdade de executar o programa significa a liberdade para qualquer tipo de pessoa física ou jurídica utilizar o software em qualquer tipo de sistema computacional, para qualquer tipo de trabalho em geral e finalidade, sem que seja necessário comunicar sobre isso com o desenvolvedor ou a qualquer outra entidade em especial.

A liberdade de redistribuir cópias deve incluir formas binárias ou executáveis do programa, assim como o código-fonte, tanto para as versões modificadas.

A liberdade de acesso ao código-fonte do programa. Portanto, acesso ao código-fonte é uma condiçao necessária ao software livre ~\cite{}.%GNU Operating System http://www.gnu.org/philosophy/free-sw.en.html acesso novembro 2013.