\chapter{Métodos Ágeis}

\section{Princípios Ágeis}

%
\begin{enumerate}
\item Nossa maior prioridade é satisfazer o cliente, através da entrega adiantada e contínua de software de valor.
\item Aceitar mudanças de requisitos, mesmo no fim do desenvolvimento. \item Processos ágeis se adéquam a mudanças, para que o cliente possa tirar vantagens competitivas.
\item Entregar software funcionando com frequência, na escala de semanas até meses, com preferência aos períodos mais curtos.
\item Pessoas relacionadas a negócios e desenvolvedores devem trabalhar em conjunto e diariamente, durante todo o curso do projeto.
\item Construir projetos ao redor de indivíduos motivados. Dando a eles o ambiente e suporte necessário, e confiar que farão seu trabalho.
\item O Método mais eficiente e eficaz de transmitir informações para, e por dentro de um time de desenvolvimento, é através de uma conversa cara a cara.
\item Software funcional é a medida primária de progresso.
\item Processos ágeis promovem um ambiente sustentável. Os patrocinadores, desenvolvedores e usuários, devem ser capazes de manter indefinidamente, passos constantes.
\item Contínua atenção à excelência técnica e bom design, aumenta a agilidade.
\item Simplicidade: a arte de maximizar a quantidade de trabalho que não precisou ser feito.
\item As melhores arquiteturas, requisitos e designs emergem de times auto-organizáveis.
\item Em intervalos regulares, o time reflete em como ficar mais efetivo, então, se ajustam e otimizam seu comportamento de acordo.
\end{enumerate}