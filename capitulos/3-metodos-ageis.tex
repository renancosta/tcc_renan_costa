\chapter{Métodos Ágeis}
\label{cap-metodos-ageis}

Métodos ágeis (AM) é uma coleção de metodologias baseada na prática para 
modelagem efetiva de sistemas baseados em software. É uma filosofia onde muitas metodologias se encaixam.

%
As metodologias ágeis aplicam uma coleção de práticas, guiadas por princípios e valores que podem ser aplicados por profissionais de software no dia a dia~\cite{}%Procurar referencia.

\section{Metodologia Ágil}
\label{metodologia-agil}

A expressão "Metodologias Ágeis" tornou-se popular em 2001 quando dezessete especialistas em processos de desenvolvimento de software representando diversos métodos como Scrum ~\cite{}, %Schwaber e Beedle 2002
Extreme Programming (XP) e outros, estabeleceram princípios comuns compartilhados por todos esses métodos. Foi então criada a Aliança Ágil e o estabelecimento do Manifesto Ágil ~\cite{} %Manifesto ágil 2013.
Os conceitos chave do Manifesto Ágil são:
\textbf{Indivíduos e interações} ao invés de processos e ferra-
mentas.
\textbf{Software executável} ao invés de documentação.
\textbf{Colaboração do cliente} ao invés de negociação de con-
tratos.
\textbf{Respostas rápidas a mudanças} ao invés de seguir pla-
nos.
O Manifesto Ágil não discrimina processos, ferramentas, documentação, negociação de contratos ou o planejamento, mas simplesmente mostra que eles têm importância secundária quando comparado com os indivíduos, interações, software executável, participação do cliente e feedback rápido a mudanças e alterações.

%
Tory Diba em uma revisão sistemática elenca alguns dos principais métodos ágeis presente em 2008 e trás uma breve descrição de cada um, o que ajuda a observar caractériscas das comunidades ágeis já que os principais especialistas desses métodos ajudaram a escrever o manifesto ágil.\cite{}%citar Diba revisão

\begin{description}

\item [Crystal Clear]
%
centra-se na comunicação de pequenas equipes de
desenvolvimento de software que não possui ciclo de vida crítico.
%
Seu desenvolvimento tem sete características: entrega frequente,
melhoria reflexivo, comunicação osmótica, segurança pessoal, foco,
fácil acesso a usuários experientes e os requisitos para o ambiente técnico.

\item [Dynamic software development method (DSDM)]
%
divide projetos em três fases: pré-projeto, o ciclo de vida do projeto  e pós projeto. Nove princípios subjacentes ao DSDM: o envolvimento do usuário, capacitando da equipe do projeto,  a entrega frequente, atender às necessidades de negócios atuais,  desenvolvimento iterativo e incremental, permitir mudanças,  o escopo de alto nível a ser fixados antes do início do projeto,  testar todo o ciclo de vida e eficiente e eficaz comunicação.
            
\item [Feature-driven development]
%
combina desenvolvimento model-driven e ágil, com ênfase  em modelo inicial de objeto, a divisão do trabalho em funções  e design iterativo para cada recurso. Afirma ser adequado para  o desenvolvimento de sistemas críticos. Uma iteração de um  recurso é composto de duas fases: concepção e desenvolvimento
       
\item [Lean software development]
%
uma adaptação de princípios de produção de carne magra e,  em particular, o sistema de produção Toyota para desenvolvimento  de software. Consiste em sete princípios: eliminar o desperdício,  aumentar a aprendizagem, decidir o mais tarde possível,  entregar o mais rápido possível, capacitar a equipe,  construir integridade, e ver o todo
                    
\item [Scrum]
%
centra-se na gestão de projetos em situações em que é difícil  planejar o futuro, com mecanismos de '' controle de processos  empíricos ", onde loops de feedback constituem o elemento central.  Software é desenvolvido por uma equipa de auto-organização  em incrementos (chamado sprints), começando com o planejamento  e terminando com uma retrospectiva. Recursos a serem implementadas  no sistema estão registrados em um backlog . Em seguida,  o proprietário do produto decide quais itens do backlog deve ser  desenvolvido da seguinte sprint. Coordenar membros da equipe  de seu trabalho em uma reunião diária de stand-up. Um membro  da equipe, o scrum master, é responsável pela resolução de  problemas que impedem a equipe de trabalho de forma eficaz.
 
\item [Extreme programming (XP)]
concentra-se nas melhores práticas para o desenvolvimento.  Consiste em doze práticas: o jogo de planejamento, pequenos lançamentos,  metáfora, design simples, testes, refatoração, programação em pares,  propriedade coletiva, integração contínua, 40 h semana, os clientes no local,  e os padrões de codificação. A revista XP2 consiste nas seguintes  ''práticas primárias": sentar-se juntos, toda a equipe, trabalho informativo,  o trabalho energizado, programação em pares, histórias, ciclo semanal,  ciclo trimestral, slack, construção de 10 minutos, integração contínua,  testes primeiro que programação e design incremental.  Há também 11 ``corollary practices''.
\end{description}

\section{Princípios Ágeis}
\label{principios-ageis}
O manifesto ágil define 12 princípios que devem ser seguidos \ref{introducao}:
%
\begin{enumerate}
\item Nossa maior prioridade é satisfazer o cliente, através da entrega adiantada e contínua de software de valor.
\item Aceitar mudanças de requisitos, mesmo no fim do desenvolvimento. Processos ágeis se adéquam a mudanças, para que o cliente possa tirar vantagens competitivas.
\item Entregar software funcionando com frequência, na escala de semanas até meses, com preferência aos períodos mais curtos.
\item Pessoas relacionadas a negócios e desenvolvedores devem trabalhar em conjunto e diariamente, durante todo o curso do projeto.
\item Construir projetos ao redor de indivíduos motivados. Dando a eles o ambiente e suporte necessário, e confiar que farão seu trabalho.
\item O Método mais eficiente e eficaz de transmitir informações para, e por dentro de um time de desenvolvimento, é através de uma conversa cara a cara.
\item Software funcional é a medida primária de progresso.
\item Processos ágeis promovem um ambiente sustentável. Os patrocinadores, desenvolvedores e usuários, devem ser capazes de manter indefinidamente, passos constantes.
\item Contínua atenção à excelência técnica e bom design, aumenta a agilidade.
\item Simplicidade: a arte de maximizar a quantidade de trabalho que não precisou ser feito.
\item As melhores arquiteturas, requisitos e designs emergem de times auto-organizáveis.
\item Em intervalos regulares, o time reflete em como ficar mais efetivo, então, se ajustam e otimizam seu comportamento de acordo.
\end{enumerate}

Os princípios ágeis devem ser vividos pela equipe e é o principal motivador e base para o desenvolvimento ágil. As práticas ágeis adotadas serão de acordo com o método escolhido e ajudarão a obter os objetivos propostos pelo método, mas são nos princípios ágeis isso sempre deve ser fundamentado de acordo com o manifesto ágil.