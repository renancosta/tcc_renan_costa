\chapter{Usabilidade}
\label{cap-usabilidade}

\section{Terminologias}
\label{sec-terminologias}

% No TCC 2, talvez, terminologia deve ser um capítulo inicial do trabalho.

O termo usabilidade de modo geral pode ser escrito como a facilidade com a qual
um equipamento ou programa pode ser usado. Esse termo dentro da computação foi
diversas vezes refinado como nas ISO 9126, 12119, 9241, 14598 e, por fim, na ISO
25010, que define como uma medida pela qual um produto pode ser usado por
usuários específicos para alcançar metas específicas com eficácia, eficiência e
satisfação em um contexto específico de uso~\cite{}.
%TODO: colocar referência da ISO.
%
A usabilidade não é uma qualidade intrínseca de um sistema, é dependente de um
acordo entre as características de sua interface e as características de seus
usuários na busca de determinados objetivos e situação de uso~\cite{}.
%TODO: referência para Cybis, 2010]
%
Por esse motivo uma interface que pode ser considerada satisfatória para
determinado grupo de usuários pode ser inviabilizada por outros, como usuários
experientes \textit{versus} novatos, além de uma percepção diferente dependendo
do ambiente onde esse sistema se encontra, um computador lento \texitt{versus}
computador rápido.
%
Dessa forma, podemos definir que a usabilidade é um acordo entre interface,
usuário, tarefa e ambiente. Baseado nesta definição é que pautaremos nossas
discussões e estudos de caso neste trabalho.
%
Dentro dessa necessidade de se garantir que sistemas e dispositivos estejam
adaptados à maneira como o usuário pensa, comporta-se e trabalha, entra o
conceito de ergonomia.
%
Tal conceito surgiu logo após a II Guerra Mundial, como consequência do trabalho
interdisciplinar realizado por diversos profissionais, tais como engenheiros,
fisiologistas e psicólogos, durante a guerra~\cite{}.
%TODO: referenciar Lida, 2005
%
Há algumas definições formais para o termo ``ergonomia'', de acordo com a
\textit{Ergonomics Society}, a Associação Brasileira de Ergonomia e a 
\textit{International Ergonomics Association}.
%
Para este trabalho, adotamos a definição da Associação Brasileira de Ergonomia,
que a conceitua como o estudo das interações das pessoas com a tecnologia, a
organização e o ambiente, objetivando intervenções e projetos que visem
melhorar, de forma integrada e não-dissociada, a segurança, o conforto, o
bem-estar e a eficácia das atividades humanas \cite{}.
%TODO: referência para o texto da ABE.
%
Neste contexto, a questão que norteia este trabalho é como pode-se avaliar,
entender, verificar, observar a interface de uma aplicação em determinado
contexto ou sistema?
%
Para respondermos essa questão, mapeamos as definições de alguns especialistas
em usabilidade e ergonomia, que estabeleceram critérios, regras e princípios
para nortear essa necessidade.

\begin{itemize}
\item Jakob Nielsen, em seu livro \textit{Usability Engineering} , propõe um
conjunto de dez heurísticas de usabilidade~\cite{}:
%TODO: referência para o livro Usability Engineering

    \begin{itemize} 

    \item Viabilidade do estado do sistema;

    \item Mapeamento entre o sistema e o mundo realizada;

    \item Liberdade e controle ao usuário;

    \item Consistência e padrões;

    \item Prevenção de erros;

    \item Reconhecer em vez de relembrar;

    \item Flexibilidade e eficiência de uso;

    \item Design estético e minimalista;

    \item Suporte para o usuário reconhecer, diagnosticar e recuperar erros;

    \item Ajuda e documentação.

    \end{itemize}

\item Ben Shneiderman, em seu  \textit{livro Designing The User Interface},
propõe, o que ele denominou de ``oito regras de ouro'':

    \begin{itemize} 

    \item Perseguir a consistência;

    \item Fornecer atalhos;

    \item Fornecer feedback informativos;

    \item Marcar o final dos diálogos;

    \item Fornecer prevenção e manipulação simples de erros;

    \item Permitir o cancelamento das ações;

    \item Fornecer controle e iniciativa ao usuário;

    \item Reduzir a carga de memória de trabalho.

    \end{itemize}

\item Christian Bastien e Dominique Scapin definiram 8 critérios ergonômicos:
%TODO: completar (Primeiro e segundo nome dos autores para ficar igual aos
%      anteriores) e colocar as referências

    \begin{itemize}

    \item Condução;

    \item Carga de trabalho;

    \item Controle;

    \item Adaptabilidade;

    \item Gestão de erros;

    \item Coerência;

    \item Significado dos códigos;

    \item Senominações e Compatibilidade.

    \end{itemize}

\end{itemize}

Portanto, baseado nas heurísticas e critérios listados acima,  Walter Cybis, no
livro Ergonomia e Usabilidade, propôs uma tabela que relaciona todas essas
definições, conforme aprensentado na Tabela \ref{tabela-walter-cybis}~\cite{}.
%TODO: referência de  Walter Cybis

\section{Práticas de Usabilidade Ágeis}
%TODO

\section{Usabilidade em Software Livre}
%TODO

\section{Métodos de Avalição}

Uma  vez que conceituamos as práticas de usabilidade ágeis e o que 
encontramos sobre usabilidade em projetos de software livre, nesta seção
detalharemos o método escolhido para...
%TODO: fazer uma introdução da subseção...

\subsection{Avaliação heurísticas}
Uma avaliação heurística representa um julgamento de valor sobre as qualidades
ergonômicas das Interfaces Humano-Computador. Essa avaliação é realizada por
especialistas em ergonomia, com base em sua experiência e competência no
assunto~\cite{}.
%TODO: referência de [Cybis, 2010]
%
Para utilização de uma avaliação heurística serão definidos os graus de
severidade, de acordo com~\cite{}:
%TODO: referência de Jakob Nielsen proposto em 1995
\begin{enumarate}

    \item Não há consenso quanto a ser um problema de usabilidade: ...

    \item Problema cosmético: ...

    \item Problema menor: ...

    \item Problema importante de usabilidade: ...

    \item Catástrofe de usabilidade: ...

\end{enumerate}
%TODO: colocar a definição de cada um dos itens acima...

A presentação dos resultados seguirá um modelo simples similar ao que é
utilizado em desenvolvimento ágil para documentação de defeitos, elencando o
problema, a possível solução e o grau de severidade.
%TODOS: contextualizar e linkar este final com o foi dito antes... está vago.

\subsection{Inspeções por listas de verificação}
As inspeções de ergonomia por meio de listas de verificação permitem que
profissionais, não necessariamente especialistas em ergonomia, identifiquem
problemas menores e repetitivos das interfaces.
%
Nesse tipo de técnica, ao contrário das avaliações heurísticas, são mais as
qualidades explicativas da ferramenta e menos os conhecimentos implícitos dos
avaliadores que determinam as possibilidades para a avaliação~\cite{}.
%TODO: ref. Cybis, 2010
%
Através das inspeções de ergonomia será possível suprir um deficit ocasionado
pela falta de experiência do avaliador dentro de determinados contextos do
sistema que este não esteja familiarizado.
%
A ISO 9241 fornece listas de verificação de ergonomia bem definidas, porém será
utilizado as listas do laboratório LabIUtil do projeto
ErgoList~\footnote{\url{http:\\labiutil.inf.ufsc.br/ergolist/check.htm},
que fornece um serviço na Internet para aplicar aplicarmos uma avaliação
simplificada e objetiva (\textit{check-list}) e obtermos os resultados
imediatamente.
%
Com a aplicação da lista pode obter vantagens como obter conhecimentos
ergonômicos, reduzir a subjetividade normalmente associada a processos de
avaliação e sistematizar as avaliações se tratando de abrangência de componentes
a inspecionar.

%TODO tem que fechar melhor o capítulo...
