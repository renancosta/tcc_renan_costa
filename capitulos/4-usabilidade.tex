\chapter{Usabilidade}

\section{Definição}
O termo usabilidade de modo geral pode ser escrito como a facilidade com a qual um equipamento ou programa pode ser usado. Esse termo dentro da computação foi diversas vezes refinado como nas ISO 9126, 12119, 9241, 14598 ou por especialistas da usabilidade como Jakob Nielsen e o mais recente na ISO 25010 que define como uma medida pela qual um produto pode ser usado por usuários específicos para alcançar metas específicas com eficácia, eficiência e satisfação em um contexto específico de uso.

A usabilidade não é uma qualidade intrínseca de um sistema, ela é dependente de um acordo entre as características de sua interface e as características de seus usuários na busca de determinados objetivos e situação de uso [Cybis, 2010].Por esse motivo uma interface que pode ser considerada satisfatória para determinado grupo de usuários pode ser inviabilizada por outros, como usuários experientes x novatos, além de uma percepção diferente dependendo do ambiente onde esse sistema se encontra, um computador lento x computador rápido. Podemos dizer então que a usabilidade é um acordo entre interface, usuário, tarefa e ambiente.

Dentro dessa necessidade de se garantir que sistemas e dispositivos estejam adaptados à maneira como o usuário pensa, comporta-se e trabalha, entra o conceito de ergonomia. Ela surgiu logo após a II Guerra Mundial, como consequência do trabalho interdisciplinar realizado por diversos profissionais, tais como engenheiros, fisiologistas e psicólogos, durante a guerra [Lida, 2005]. A definição do termo é amplamente discutida e diversas associações trazem sua própria definição, mas dentre essas as que mais se destacam são:

“Ergonomia é o estudo do relacionamento entre o homem e seu trabalho, equipamento, ambiente e particularmente, a aplicação dos conhecimentos de anatomia, fisiologia e psicologia na solução dos problemas que surgem desse relacionamento” 
Ergonomics Society, definição mais antiga do termo.

“Entende-se por Ergonomia o estudo das interações das pessoas com a tecnologia, a organização e o ambiente, objetivando intervenções e projetos que visem melhorar, de forma integrada e não-dissociada, a segurança, o conforto, o bem-estar e a eficácia das atividades humanas”
Associação Brasileira de Ergonomia

Em 2000 a International Ergonomics Association aprovou uma definição conceituando a ergonomia e suas especializações que diz Ergonomia é a disciplina científica, que estuda as interações entre os seres humanos e outros elementos do sistema, e a profissão que aplica teorias, princípios, dados e métodos, a projetos que visem otimizar o bem-estar humano e o desempenho global de sistemas.

Os praticantes da ergonomia são denominados de ergonomistas e realizam o planejamento, projeto, avaliação das tarefas, prodtos, ambiente e sistemas. Esses trabalham em domínios especializados, abordando certas características específicas do sistema que são:

Ergonomia Física – Ocupa-se das características da anatomia humana, antropometria, fisiologia e biomecânica, relacionados com a atividade física;
Ergonomia Cognitiva – Ocupa-se dos processos mentais, como a percepção, memória, raciocínio e resposta motora, relacionados com as interações entre as pessoas e outros elementos de um sistema;
Ergonomia Organizacional – Ocupa-se da otimização dos sistemas sócio-técnicos, abrangendo as estruturas organizacionais, políticas e processos. [Lida, 2005]

Como então se pode avaliar, entender, verificar, observar a interface de uma aplicação em determinado contexto ou sistema? Dentro dessa questão alguns especialistas, papas da usabilidade e ergonomia definiram heurísticas, critérios, regras e princípios para nortear essa necessidade.

Jakob Nielsen em seu livro Usability Engineering de 1994, propõe um conjunto de dez heurísticas de usabilidade:
\begin{itemize} 
\item Viabilidade do estado do sistema;
\item Mapeamento entre o sistema e o mundo realizada;
\item Liberdade e controle ao usuário;
\item Consistência e padrões;
\item Prevenção de erros;
\item Reconhecer em vez de relembrar;
\item Flexibilidade e eficiência de uso;
\item Design estético e minimalista;
\item Suporte para o usuário reconhecer, diagnosticar e recuperar erros;
\item Ajuda e documentação.
\end{itemize}

Ben Shneiderman em seu livro Designing the user interface propõe oito regras de ouro:
\begin{itemize} 
\item Perseguir a consistência;
\item Fornecer atalhos;
\item Fornecer feedback informativos;
\item Marcar o final dos diálogos;
\item Fornecer prevenção e manipulação simples de erros;
\item Permitir o cancelamento das ações;
\item Fornecer controle e iniciativa ao usuário;
\item Reduzir a carga de memória de trabalho.
\end{itemize}

Baseado nessas heurísticas, junto aos 8 critérios (Condução, Carga de trabalho, Controle, Adaptabilidade, Gestão de erros, Coerência, Significado dos códigos e denominações e Compatibilidade) de Bastien e Scapin, regras e princípios de ergonomia foi possível criar uma tabela trazida por Walter Cybis no livro Ergonomia e Usabilidade que relaciona todas essas definições.

\section{Práticas de Usabilidade Ágeis}

\section{Usabilidade em Software Livre}

\section{Métodos de Avalição (Detalhamento do Método Escolhido)}
Avaliação heurísticas

Uma avaliação heurística representa um julgamento de valor sobre as qualidades ergonômicas das Interfaces Humano-Computador. Essa avaliação é realizada por especialistas em ergonomia, com base em sua experiência e competência no assunto [Cybis, 2010].
Para utilização de uma avaliação heurística serão definidos os graus de severidade de acordo com Jakob Nielsen proposto em 1995
\begin{itemize}
\item 0 -não há consenso quanto a ser um problema de usabilidade
\item 1 -problema cosmético
\item 2 -problema menor
\item 3 -problema importante de usabilidade - corrigir
\item 4 -Catástrofe de usabilidade - imperativo corrigir!
\end{itemize}

A presentação dos resultados seguirá um modelo simples similar ao que é utilizado em desenvolvimento ágil para documentação de defeitos, elencando o problema, a possível solução e o grau de severidade.

Inspeções por listas de verificação

As inspeções de ergonomia por meio de listas de verificação permitem que profissionais, não necessariamente especialistas em ergonomia, identifiquem problemas menores e repetitivos das interfaces. Nesse tipo de técnica, ao contrário das avaliações heurísticas, são mais as qualidades explicativas da ferramenta e menos os conhecimentos implícitos dos avaliadores que determinam as possibilidades para a avaliação [Cybis, 2010].
Através das inspeções de ergonomia será possível suprir um deficit ocasionado pela falta de experiência do avaliador dentro de determinados contextos do sistema que este não esteja familiarizado.
A ISO 9241 fornece listas de verificação de ergonomia bem definidas, porém será utilizado as listas do laboratório LabIUtil do projeto ErgoList: http://www.labiutil.inf.ufsc.br/ergolist/check.htm que fornece um ambiente online e simplificado para aplicar o checklist e obter resultado imediato.
Com a aplicação da lista pode obter vantagens como obter conhecimentos ergonômicos, reduzir a subjetividade normalmente associada a processos de avaliação e sistematizar as avaliações se tratando de abrangência de componentes a inspecionar.