\chapter{Estudo de Caso: projeto Mezuro}
\label{cap-estudo-caso}

O trabalho de pesquisa desenvolvido durante o capítulo \ref{cap-usabilidade}, onde se levantou técnicas de usabilidade ágeis que poderiam ser aplicadas na comunidade de software livre conforme seção \ref{técnicas-usabilidade-ageis} e depois a seleção de uma dessas técnicas a fim de realizar sua descrição e detalhamento na seção \ref{usabilidade-sl} tinha como objetivo definir a forma que seria integrado a usabilidade em uma equipe de desenvolvimento em um contexto de software livre. Detalhar essa forma de trabalho se faz necessário, pois esses conceitos e métodos serão apresentados e aplicados em uma ferramenta real que será utilizada como estudo de caso.

%
YIN (1989)%YIN, Robert K. - Case Study Research - Design and Methods. Sage Publications Inc., USA, 1989. 
diz que "o estudo de caso é uma inquirição empírica que investiga um fenômeno contemporâneo dentro de um contexto da vida real, quando a fronteira entre o fenômeno e o contexto não é claramente evidente e onde múltiplas fontes de evidência são utilizadas".
Validar a pesquisa através de sua aplicação trás vantagens como:
\begin{itemize}
\item Produção de artefatos descritivos suficientemente rico para permitir reinterpretações subsequentes.
\item Relacionamento da teoria com a prática
\item Melhor percepção através de exemplos específicos, acontecimentos ou experiências
\end{itemize}

%
Segundo YIN (1989), para um estudo de caso deve se definir:
\begin{itemize}
\item Uma visão geral do projeto do estudo de caso
\item Os procedimentos que serão aplicados, abordados
\item As questões do estudo de caso que o investigador deve ter em mente, os locais, as fontes de informação, os formulários para o registro dos dados e as potenciais fontes de informação para cada questão
\item Um guia para o relatório do Estudo do Caso
\end{itemize}

%
A definição dessa estrutura se dá de forma automática devido ao modelo adotado de abordagem da usabilidade dentro do projeto facilitando a aplicação do método.

A ferramenta definida para o estudo de caso foi o software livre Mezuro, uma plataforma de monitoramento de código-fonte. A escolha desta ferramenta se deve ao fato que, ao ser concebida, um trabalho similiar foi realizado pela pesquisadora Ana Paula Oliveira dos Santos, porém a ferramenta se tratava de um plugin de outra plataforma e tornou-se necessário uma migração para uma ferramenta standalone. Com essa troca muito foram as mudanças e os aspectos de usabilidade passaram a necessitar de um novo planejamento.Outro fator importante foi a facilidade de estar inserido junto a equipe de desenvolvimento, ficando assim mais próximo do projeto.

\section{Mezuro}
\label{mezuro}
O Mezuro é uma plataforma para analisar e compreender as métricas, bem como, aplicar os conceitos de código limpo, medição de software e atratividade do software livre. Ele é um software livre sob licença AGPL~\footnote{GNU Affero General Public License (AGPL): \url{gnu.org/licenses/agpl.html}}. Mezuro usa Kalibro Metrics , que é um serviço web que pode se conectar a diferentes tipos de repositórios de código-fonte , baixar projetos de software e executar muitos coletor métrica integrada e ferramentas de calculadora automaticamente. Kalibro é licenciado LGPL~\footnote{GNU Lesser General Public License (LGPL):\url{gnu.org/licenses/lgpl.html}}~\cite{}.%Artigo mezuro
É uma aplicação Web para que os líderes e desenvolvedores de projetos de software livre possam monitorar características de código-fonte, através de métricas de acoplamento, coesão, tamanho, encapsulamento, entre outras. Isso proporciona um acompanhamento do quanto o software está crescendo e se tornando mais complexo em relação a ele mesmo e à média dos projetos avaliados pelo Mezuro~\cite{}%Ana Paula mestrado
.

A plataforma Mezuro foi desenvolvida no contexto do projeto QualiPSo (\textit{Trust and Quality in Open Source Systems}).
O projeto integrado Qualipso (Quality Plataform for Open Source) busca definir e implementar tecnologias, procedimentos, leis e políticas com o objetivo de potencializar as práticas de desenvolvimento de software livre, tornando-as confiáveis, reconhecidas e estabelecidas na indústria.
Para viabilizar o projeto e a sustentação do software livre como uma solução confiável para a indústria, criou-se um consórcio formado por colaboradores de diferentes origens: França, Itália, Brasil, Espanha, China, Alemanha e Escócia~\cite{}%Qualipso http://qualipso.icmc.usp.br/ acessado em: novembro 2013
.

Para o desenvolvimento do Mezuro realizou-se um estudo de plataformas correlatas a fim de melhor situar as resoluções já existentes~\cite{}%Paulo Meirelles
. As ferramentas estudadas foram:

\begin{description}
\item[FLOSSMetrics (Free /Libre Open Source Software Metrics)]
é um projeto que utiliza metodologias e ferramentas existentes para fornecer um grande banco de dados com informações sobre o desenvolvimento de software livre
\item[Ohloh]
é uma plataforma que oferece um conjunto de serviços web e um sistema web para comunidade de software livre, que visa prover uma visão geral de evolução dos projetos de software livre em desenvolvimento
\item[Qualloss (Quality in Open Source Software)]
é uma metodologia para automatizar a medição da qualidade de projetos de software livre, usando ferramentas para analisar o código-fonte e as informações dos repositórios dos projetos
\item[SQO-OSS (Software Quality Assessment of Open Source Software)]
fornece um conjunto de ferramentas para análise e avaliação comparativa de projetos de software livre
\item[QSOS]
é uma metodologia baseada em quatro etapas: definição de referência utilizada; avaliação de software; qualificação dos usuários em contexto específico; seleção e comparação de software
\item[FOSSology]
(Advancing open source analysis and development) é um projeto que fornece um banco de dados gratuito com informações sobre licenças de software livre
\item[HackyStat]
é um ambiente para visualização, análise e interpretação do processo de desenvolvimento de software e dados do produto de software
\end{description}

\subsection{Mezuro Plugin}
\label{mezuro-plugin}

\subsection{Mezuro Standalone}
\label{mezuro-standalone}