\chapter{Estudo de Caso: projeto Mezuro}
\label{cap-estudo-caso}

O trabalho de pesquisa desenvolvido durante o capítulo \ref{cap-usabilidade}, onde se levantou técnicas de usabilidade ágeis que poderiam ser aplicadas na comunidade de software livre conforme seção \ref{técnicas-usabilidade-ageis} e depois a seleção de uma dessas técnicas a fim de realizar sua descrição e detalhamento na seção \ref{usabilidade-sl} tinha como objetivo definir a forma que seria integrado a usabilidade em uma equipe de desenvolvimento em um contexto de software livre. Detalhar essa forma de trabalho se faz necessário, pois esses conceitos e métodos serão apresentados e aplicados em uma ferramenta real que será utilizada como estudo de caso.

%
YIN (1989)%YIN, Robert K. - Case Study Research - Design and Methods. Sage Publications Inc., USA, 1989. 
diz que "o estudo de caso é uma inquirição empírica que investiga um fenômeno contemporâneo dentro de um contexto da vida real, quando a fronteira entre o fenômeno e o contexto não é claramente evidente e onde múltiplas fontes de evidência são utilizadas".
Validar a pesquisa através de sua aplicação trás vantagens como:
\begin{itemize}
\item Produção de artefatos descritivos suficientemente rico para permitir reinterpretações subsequentes.
\item Relacionamento da teoria com a prática
\item Melhor percepção através de exemplos específicos, acontecimentos ou experiências
\end{itemize}

%
Segundo YIN (1989), para um estudo de caso deve se definir:
\begin{itemize}
\item Uma visão geral do projeto do estudo de caso
\item Os procedimentos que serão aplicados, abordados
\item As questões do estudo de caso que o investigador deve ter em mente, os locais, as fontes de informação, os formulários para o registro dos dados e as potenciais fontes de informação para cada questão
\item Um guia para o relatório do Estudo do Caso
\end{itemize}

%
A definição dessa estrutura se dá de forma automática devido ao modelo adotado de abordagem da usabilidade dentro do projeto facilitando a aplicação do método.

A ferramenta definida para o estudo de caso foi o software livre Mezuro, uma plataforma de monitoramento de código-fonte. A escolha desta ferramenta se deve ao fato que, ao ser concebida, um trabalho similiar foi realizado pela pesquisadora Ana Paula Oliveira dos Santos, porém a ferramenta se tratava de um plugin de outra plataforma e tornou-se necessário uma migração para uma ferramenta standalone. Com essa troca muito foram as mudanças e os aspectos de usabilidade passaram a necessitar de um novo planejamento.Outro fator importante foi a facilidade de estar inserido junto a equipe de desenvolvimento, ficando assim mais próximo do projeto.

\section{Mezuro}
\label{mezuro}

\subsection{Mezuro Plugin}
\label{mezuro-plugin}

\subsection{Mezuro Standalone}
\label{mezuro-standalone}