\chapter{Considerações Finais}
\label{cap-consideracoes}

%
Projetos de grande porte que geralmente possuem o patrocínio de grandes empresas, a aplicação das técnicas de usabilidade são melhor observadas, pois estas podem contar com uma equipe formada por engenheiros de usabilidade e/ou especialistas em usabilidade. Já em projetos menores, raramente alguma prática de usabilidade é utilizada. Contudo, a maioria dos projetos de software livre são desenvolvidos por equipes pequenas, compostas geralmente, apenas por desenvolvedores e que não possuem especialistas em usabilidade ou áreas relacionadas como participantes do projeto. Ainda assim, a usabilidade é importante neste contexto.

Entender então quais as técnicas de usabilidade que poderiam ser aplicadas nesta comunidade se fez necessário realizando assim uma revisão sistemática com o foco de levantar as técnicas de usabilidade em projetos ágeis que possam ser aplicadas no desenvolvimento de software livre.

%
Algumas destas técnicas identificadas podem ser aplicadas sem nenhum grande impacto em ambientes de desenvolvimento de software livre que utilizem de práticas ágeis, onde seja possível a realização destas presencialmente. Porém mesmo em um desenvolvimento distribuído é possível a aplicação das técnicas sem grandes modificações com apoio de ferramentas de comunicação via internet.

%
As técnicas foram retiradas de publicações que traziam a aplicação de técnicas de usabilidade dentro do ambiente de desenvolvimento ágil ocasionando o menor impacto possível nesse ambiente, o que facilita a compreensão e utilização dessas técnicas por parte da equipe de desenvolvimento.

%
Apesar de diversas técnicas encontradas uma foi escolhida e detalhada para poder se aplicar em um estudo de caso e observar os resultados obtidos junto a equipe e no produto final. A ferramenta Mezuro foi escolhida como estudo de caso, pois se encontrava em um novo ciclo de desenvolvimento voltado na evolução da ferramenta e em uma mudança de plataforma, o que possibilitava a inserção do ciclo de usabilidade definido na seção \ref{mezuro-standalone}.

%
A evolução do Mezuro não foi motivada por um pretexto isolado. Ela surgiu da necessidade de portar o mezuro plugin para uma plataforma independente do Noosfero. Isso ocorreu devido à limitações ocasionadas pelo Noosfero. Uma dessas limitações deve-se ao fato de seu desenvolvimento ser baseado em uma versão antiga do rails com pouco suporte pela comunidade limitado as funções dessa versão e as funcionalidades e visão de rede social que não são necessárias ao Mezuro. Com essa mudança a equipe adquiriu uma liberdade em sua tomada de decisões, por não estar vinculada a comunidade Noosfero.

\section{Cronograma}

Para o curto e médio prazo, no escopo deste trabalho, há atividades planejadas para o futuro do Mezuro. Até agora, o Mezuro evoluiu em sua arquitetura, migrando de plugin para aplicação independente com o Rails 4. Na segunda fase deste trabalho, vamos colaborar com novas funcionalidades, que também demandará uma etapa de pesquisa complementar ao que fizemos até agora. As atividades planejadas são:

\begin{enumerate}
\item Definir metodologia de pesquisa
\item Complemento das pesquisas sobre software livre e sua relação com a comunidade ágil
\item Pesquisa sobre a contribuição e resultados da usabilidade em comunidades de software livre;
\item Incorporar junto a equipe o ciclo de usabilidade;
\item Refatorar funcionalidades implementadas no Mezuro de acordo com a análise realizada com os métodos de avaliação;
\item Colaboração com a finalização da migração das funcionalidades do Mezuro Plugin para o Mezuro standalone visando os aspectos de usabilidade definidos no Capítulo \ref{mezuro-standalone};
\item Coletar, analisar e documentar os resultados da aplicação da técnica selecionada 
\end{enumerate}

\begin{table}[H]
\begin{center}
    \begin{tabular}{ | l | l | l | l | l | l | l | l |}
    \hline
    Atividade & Dez 2013 & Jan 2014 & Fev 2014 & Mar 2014 & Abr 2014 & Mai 2014 & Jun 2014 \\ \hline
    1 & • & • &   &   &   &   &   \\ \hline
    2 &   & • & • & • &   &   &   \\ \hline
    3 & • & • &   &   &   &   &   \\ \hline
    4 & • & • & • &   &   &   &   \\ \hline
    5 & • & • & • &   &   &   &   \\ \hline
    6 & • & • & • & • &   &   &   \\ \hline
    7 &   &   &   &   & • & • & • \\ \hline
    \end{tabular}
    \caption{Cronograma para atividades do TCC2}
    \label{tab-cronograma}
\end{center}
\end{table}

Essa primeira fase do projeto contribuiu em uma definição e pesquisa do ciclo de usabilidade que será aplicado no estudo de caso, visando a melhor estratégia junto a equipe para o desenvolvimento na comunidade de software livre. Uma primeira iteração foi realizada para avaliar o que já existe e a forma que foi implementada e com isso gerado protótipos de tela para uma proposta de melhoria.  A segunda fase implementará todo o ciclo de usabilidade junto a equipe de forma participativa no desenvolvimento e ao fim a coleta, análise e documentação dos resultados da aplicação da técnica selecionada.