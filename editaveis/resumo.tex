\begin{resumo}

Muitos dos projetos de software livre não possuem práticas para melhoria da usabilidade em seu processo de desenvolvimento, o que pode influenciar em sua adoção.
%
O desenvolvimento de software livre pode ser considerado um método ágil do ponto de vista da engenharia de software.
%
O objetivo deste trabalho foi aplicar tarefas de usabilidades através de técnicas de usabilidade ágeis em comunidades de software livre.
%
Para isso, foi empregado uma revisão sistemática e bibliográfica para levantamento da usabilidade em software livre e um estudo de caso a fim de utilizá-las em um contexto real levantado assim os resultados através de métodos de avaliação.

 \vspace{\onelineskip}
    
 \noindent
 \textbf{Palavras-chaves}: usabilidade, software livre, técnicas de usabilidade, métodos ágeis.
\end{resumo}
